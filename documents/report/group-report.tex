\documentclass[12pt,a4paper,twoside]{report}

\usepackage{cite}
\usepackage{graphicx}
\usepackage{url}
\usepackage{microtype}
\usepackage{titlesec}
\usepackage[titletoc]{appendix}
\usepackage{lipsum}

\linespread{1.3}
\titleformat{\chapter}{\large\bf\centering}{CHAPTER \MakeUppercase{\thechapter}.}{1em}{}
\titleformat{\section}{\normalfont\bf}{{\thesection}}{1em}{}
\titleformat{\subsection}{\normalfont\bf}{{\thesubsection}}{1em}{}

\begin{document}
\pagenumbering{roman}
\begin{titlepage}
\begin{center}

\textsc{\LARGE CS407 Group Report}\\[1.5cm]
\vspace{1.5cm}

\hrule
\vspace{0.2cm}
\textsc{\LARGE A framework for autonomous drone networks}\\
\vspace{0.2cm}
\hrule

\vspace{1.5cm}
\noindent
\begin{minipage}{0.4\textwidth}
	\begin{flushleft} \large
		\emph{Authors:}\\
		William \textsc{Seymour}, \\ Jon \textsc{Gibson}, \\ Alex \textsc{Henson}, \\ Ben \textsc{De Ivey}
	\end{flushleft}
\end{minipage}%
\begin{minipage}{0.4\textwidth}
	\begin{flushright} \large
		\emph{Supervisor:} \\
		Dr.~Arshad \textsc{Jhumka}
	\end{flushright}
\end{minipage}

\vfill
\large Department of Computer Science\\
\large University of Warwick\\
\large Summer 2016\\
\vfill
\includegraphics[width=0.50\textwidth]{img/dcslogo.png}~\\[1cm]
\end{center}
\end{titlepage}

\tableofcontents
\listoffigures
\listoftables

\abstract{\lipsum[1]}
\pagenumbering{arabic}

\chapter{Opening}
	\section{Key Words}
	Autonomous Drones, Sensor Networks, Network Simulation, Pathfinding, Physical Routing, Communications Routing
	\section{Word Count}
	The document contains 30,000 words. This number was calculated from the document source by TeXstudio.
	\section{Acknowledgements}
	We would like to thank our project supervisor Arshad Jhumka for guiding us and giving us advice on available technologies, methods and tools for wireless sensor network implementation, and for his continued support, even when we decided to change the foundational software basis of our project. Finally, we would also like to thank him for his critique during meetings and the poster presentation on our implementation of routing and research into the field.
	\section{Introduction}
	This report will provide a comprehensive analysis of the project undertaken by our group on the subject of autonomous drones in sensor networks. There will be a background summary of the key components in this field, as well as a discussion of the ongoing research, development and production being carried out. We will supply an analysis of the potential problems for which a solution can be found in drone networks, and a justification for the resulting aims and objectives of our group. The report will detail the design, implementation and testing of the solution, including considerations for the management of the project. Finally, the project outcome will be evaluated, followed by a conclusion reflecting on the success of the project and considerations for future works.

\chapter{Background}
	\section{Definitions}
	\section{Quadcopters}
	\section{Sensor Networks}
		\subsection{The Internet of Things}
	\section{Drone Networks}
	\section{Just an example of how we might break them down}
	
\chapter{Specification}
	\section{Description of the Problem}
\section{Objectives}
		\subsection{Quantitative Objectives}
		\subsection{Qualitative Objectives}
	\section{Justification}
	\section{Stakeholder Analysis}
	\section{Feasibility Study}
	\section{Requirements Identification}
		\subsection{Functional Requirements}
		\subsection{Non-functional Requirements}
	\section{Project Deliverables}
	\section{Changes From Original Specification}

\chapter{Literature Review}
	\section{Existing Solutions}
	\section{Social and Ethical Issues}
	\section{Physical Routing}
	\section{MANETS}
	\section{Drones}	

\chapter{Design}
	\section{Methodology}
	\section{System Architecture}
		\subsection{Network Simulation}
			\subsubsection{Fundamental Structure}
			\subsubsection{Environment}
			\subsubsection{Application Programming Interfaces (API)}
		\subsection{Communications Routing}
		\subsection{Physical Routing} 
		\subsection{Physical Deployment}
			\subsubsection{Libraries}


\chapter{Simulation Software}
	\section{Existing Software}
		\subsection{NS3}
		\subsection{NS2}
		\subsection{Some}
		\subsection{other}
		\subsection{stuff}
	\section{Summary of Existing Software}
	\section{Development Methodology}
	\section{Code Structure}
		\subsection{The Environment}
		\subsection{Communication Modules}
		\subsection{Drone}
		\subsection{Base Station}
	\section{Results}
	\section{Optimisation}
	\section{Review Against Original Objectives}
	\section{User Manual}

\chapter{Physical Routing}

\chapter{Communications Routing}

\chapter{Physical Deployment}
	\section{Objectives}
	\section{Equipment and Feasability}
		\subsection{Arduino}
		\subsection{RaspberryPi}
		\subsection{Something else}
	\section{Adapting the Simulator Code}
	\section{Results}
	\section{Review Against Original Objectives}

\chapter{Testing}
	\section{Unit Testing}
	\section{Client Application Testing}
	\section{Usability Testing}
	\section{Performance Testing}
	\section{System Testing}
	\section{User Acceptance Testing}
	\section{Risk Assessment}

\chapter{Project Management}
	\section{Team Structure}
	\section{Progress Tracking}
	\section{Source Control}
	\section{Time Management}
	\section{Collaboration Tools}
		\subsection{GitHub}
		\subsection{Google Drive}
	\section{Risk Management}

\chapter{Project Outcome}
	\section{Projects Deliverables}
	\section{Review Against Original Objectives}
	\section{Project Appraisal}
	\section{Future Work}

\chapter{Conclusion}


\end{document}