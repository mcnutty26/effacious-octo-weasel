\documentclass[10pt,a4paper,twoside]{report}
\usepackage{graphicx}
\usepackage{url}
\usepackage{float}
\usepackage{microtype}
\usepackage[explicit]{titlesec}
\usepackage[titletoc]{appendix}
\usepackage{lipsum}
\usepackage[normalem]{ulem}
\usepackage[hidelinks]{hyperref}
\useunder{\uline}{\ul}{}
\usepackage[hmarginratio=1:1]{geometry}
\usepackage{tikz}
\usepackage{fourier}
\usepackage{mdframed}
\usepackage{epigraph}
\usepackage{pdfpages}
\usepackage[titletoc]{appendix}
\usepackage[nottoc,numbib]{tocbibind}
\usepackage{longtable}
\usepackage[type={CC},modifier={by-sa},version={4.0}]{doclicense}
\usepgflibrary{qrr.shapes.openrectangle}

%set header and footer
\usepackage{fancyhdr}
\pagestyle{fancy}
\chead{octoDrone: Simulation and Deployment for Autonomous Drone Networks}
\lhead{}
\rhead{}
\renewcommand{\footrulewidth}{0.4pt}% Line at the footer visible
\cfoot{\thepage}

\fancypagestyle{plain}{%
  \fancyhf{}%
  \fancyfoot[C]{}%
  \renewcommand{\headrulewidth}{0pt}% Line at the header invisible
  \renewcommand{\footrulewidth}{0.4pt}% Line at the footer visible
  \cfoot{\thepage}
}

%set chapter heading colours
\definecolor{mybluei}{RGB}{0,173,239}
\definecolor{myblueii}{RGB}{63,200,244}
\definecolor{myblueiii}{RGB}{199,234,253}

%set chapter heading tikz style
\tikzset{
mynode/.style={
  rounded corners=30pt,
  shape=open rectangle,
  open rectangle fill=myblueii,
  open rectangle sides=#1,
  }
}

%set up reference links
\hypersetup{colorlinks=true,allcolors=blue}
\bibliographystyle{plain}
\usetikzlibrary{shapes.geometric, arrows, positioning}

%look for images in the image folder
\graphicspath{{img/}}

%use 1.5x line spacing
\linespread{1.5}

%set up the aside environment
\newenvironment{aside}
  {\begin{mdframed}[style=0,%
      leftline=false,rightline=false,leftmargin=2em,rightmargin=2em,%
          innerleftmargin=0pt,innerrightmargin=0pt,linewidth=0.75pt,%
      skipabove=7pt,skipbelow=7pt]\small}
  {\end{mdframed}}

\begin{document}

%use roman numbering for the front matter
\pagenumbering{roman}
\begin{titlepage}
\begin{center}

\textsc{\LARGE CS407 Group Report}\\[1.5cm]
\vspace{1.5cm}

\hrule
\vspace{0.2cm}
\textsc{\LARGE octoDrone: Simulation and Deployment for Autonomous Drone Networks}\\
\vspace{0.2cm}
\hrule

\vspace{1.5cm}
\noindent
\begin{minipage}{0.4\textwidth}
	\begin{flushleft} \large
		\emph{Authors:}\\
		Alex \textsc{Henson}, \\ Ben \textsc{De Ivey}, \\ Jonathan \textsc{Gibson}, \\ William \textsc{Seymour} 
	\end{flushleft}
\end{minipage}%
\begin{minipage}{0.4\textwidth}
	\begin{flushright} \large
		\emph{Supervisor:} \\
		Dr.~Arshad \textsc{Jhumka} \\
		\emph{Secondary Marker:} \\
		Dr.~ Andrzej \textsc{Murawski} 
	\end{flushright}
\end{minipage}
\vfill
\large Department of Computer Science\\
\large University of Warwick\\
\large Summer 2016\\
\vfill
\includegraphics[width=0.50\textwidth]{img/dcslogo.png}~\\[1cm]
\end{center}
\end{titlepage}

\newgeometry{hmarginratio=2:3}

\tableofcontents
\listoffigures
\listoftables

\abstract{Drones and quadcopters are becoming increasingly prevalent in modern society, leading to increased research and development in the field of mobile sensor networks. This project, octoDrone, aimed to create a high quality network simulator targeted at quadcopters, with the aim of being accessible to academics and students alike. Through a focus on being abstract, we have created an application which is powerful, expressive, and can be deployed to almost all currently available hardware. This ability to use the same code for simulations and deployments allows for accurate benchmarking and verification of drone programs.}

%switch back to arabic numbering
\pagenumbering{arabic}

%set the chapter headings
\titleformat{\chapter}[display]
  {\normalfont\huge\sffamily}
  {}
  {20pt}
  {%
  \begin{tikzpicture}[remember picture,overlay]
  \node[
    anchor=west,
    rectangle,
    minimum height=4cm,
    text width=\paperwidth+1in,
    xshift=-\the\dimexpr\oddsidemargin+2in\relax,
    outer sep=0pt,
    fill=myblueiii] (titlerect) {};
  \node[
    anchor=south west,
    xshift=4.5cm,
    text width=\textwidth] 
    at ([yshift=5pt]titlerect.south west) {\fontsize{30}{36}\selectfont#1};
  \node[
    mynode=nw,
    anchor=south east,
    fill=myblueii,
    inner xsep=1.5cm,
    outer sep=0pt,
    font=\color{white},
    minimum height=30pt] 
    at (current page.east|-titlerect.north)
     {\bfseries\MakeUppercase{\chaptertitlename}\ \thechapter};
  \end{tikzpicture}%
  }
\titlespacing*{\chapter}
  {0pt}{-20pt}{60pt}

\setlength\beforeepigraphskip{1.5\baselineskip}
\setlength\afterepigraphskip{2\baselineskip}
\setlength\epigraphwidth{6.8cm}
\setlength\epigraphrule{0pt}
\renewcommand\epigraphsize{\large}
\renewcommand\textflush{flushright}

\let\oldepigraph\epigraph \renewcommand\epigraph[2]{%
  \oldepigraph{\color{mybluei}\itshape #1}{#2}}
 
\chapter{Opening}
\epigraph{``Once you have tasted flight, you will forever walk the earth with your eyes turned skyward, for there you have been, and there you will always long to return'' - Leonardo da Vinci}

\section{Key Words}
	Autonomous Drones, Sensor Networks, Network Simulation, Pathfinding, Physical Routing, Communications Routing
	\section{Word Count}
	The document contains 30,000 words. This number was calculated from the document source by Texmaker.
	\section{Acknowledgements}
	We would like to thank our project supervisor Arshad Jhumka for guiding us and giving us advice on available technologies, methods and tools for wireless sensor network implementation, and for his continued support, even when we decided to change the foundational software basis of our project. Finally, we would also like to thank him for his critique during meetings and the poster presentation on our implementation of routing and research into the field.
	\section{Introduction}
	This report will provide a comprehensive analysis of the project undertaken by our group on the subject of autonomous drones in sensor networks. There will be a background summary of the key components in this field, as well as a discussion of the ongoing research, development and production being carried out. We will supply an analysis of the potential problems for which a solution can be found in drone networks, and a justification for the resulting aims and objectives of our group. The report will detail the design, implementation and testing of the solution, including considerations for the management of the project. Finally, the project outcome will be evaluated, followed by a conclusion reflecting on the success of the project and considerations for future works.

\chapter{Background}
	\lettrine[lines=2]{T}{his} section will introduce the components which will be researched into that form the basis of the project. The aim of this section is to provide the reader with definitions for keywords which will appear on numerous occasions throughout the project, as well as helping to lead into a definition of the problem space and the objectives of the project as a result. 
	\section{Drones}
		\subsection{Definition}
		Unmanned Aerial Vehicles (UAVs), also known as drones, are aircraft which are either ‘piloted’, or perform autonomously using pre-programmed flight path and objectives \cite{chriscolejimwright2010}. Consumer-level drones are typically small in size, and take the form of quadcopters, which are multi-rotor helicopters with four rotors.  These types of drones are very lightweight, and powered by batteries; power consumption is almost completely attributed to fight time, although a modified drone will be required to exhaust power on its additional parts. We will be focusing on the use of these drones for the scale of this project.
		\subsection{Sensor Capabilities}
		Drones are typically equipped with cameras, as well as additional sensors, which vary depending on the type of drone, or its purpose. In the case of military drones, sensors such as multi-spectral targeting systems, night vision, infrared imaging and GPS are an absolute must \cite{usairforce2015}. However, mounted weaponry may also be included for direct warfare, unless the drone is designed specifically for intelligence, surveillance or reconnaissance, as power consumption is a primary concern, and must be limited. Military drones are controlled via satellite from a military base, although drones may also be controlled by Wi-Fi, radio or remote. Consumer-level drones have a wide range of usages, and the sensors that they require to accommodate these tasks are largely dependent on the price. It is possible to attach a large multitude of different sensors to drones; however the primary function of an aerial (consumer-level) drone is to collect high-quality imagery, often beyond the level of detail that the human eye can process.  These types of sensors include stereoscopic, thermal imaging, near-infrared and infrared, but other sensors such as thermal sensors and proximity sensors may also be used \cite{ questuav2015}.
		\subsection{Usages}
		Consumer-level drones have become increasingly popular in the past few years as they have become more affordable, capable and reliable to use. Due to their ability to capture high-quality imagery from impossible-to-reach locations, drones are incredibly useful in areas such as real estate, to take aerial shots of properties, or as a cheap alternative to huge, expensive helicopters for capturing news, such as high speed chases \cite{josephdussault2014}. Drones can also be employed for services such as delivery; there have been several initiatives for drone-based delivery of food or packaged goods by famous companies such as Amazon and Domino's Pizza \cite{marcusfaires2015}. Another possible usage for drones is in emergency services, such as the detection of forest fires, or search and rescue. It is these areas of drone research and development which can be considered to display the strength and importance of drones, as they are able to perform dangerous tasks which humans are incapable of and/or with no physical risk to human beings themselves. Compared to other types of mobile sensing, drones offer direct control over where to sample the environment, such that they can be explicitly told where to move to.
	\section{Sensor Networks}
		\subsection{Definition}
		A wireless sensor network (WSN) is a wireless network consisting of spatially distributed autonomous devices using sensors to monitor physical or environmental conditions. These devices are referred to as nodes, which are able to communicate with each other and with a base node, commonly referred to as a gateway, which provides connectivity between itself, the nodes and the rest of the wired world \cite{ nationalinstruments2012}. Nodes may vary in size and number depending on the network, but will typically contain transceivers, a battery, an electronic circuit for interfacing with sensors and an energy source.  The ability to cooperatively pass sensor data to a main location has implications in many different industries, as well as military applications.
		\subsection{Drone Networks}
		In the context of drones, this refers to a wirelessly connected network of autonomous drones with a base station, which can distribute information or commands to the drones in the network, as well as facilitate communication between them and itself.  Given that autonomous drones are emerging as a powerful new breed of mobile sensing system which can carry rich sensor payloads with various methods of control, a collaborative network of drones has considerable potential, and can greatly extend the capabilities of traditional sensing systems \cite{lucamottola2014}. 
	\section{Network Simulation}
		\subsection{Definition}
		As the name suggests, network simulation is a technique for modelling the behaviour of a network without performing a real, physical deployment, in order to test the effectiveness of the network, and assess how the network will behave under different conditions. Therefore, a simulation refers to software that predicts the behaviour of a network, so that performance can be analysed. By emulating an existing network, unexpected problems can be addressed or prevented prior to the deployment of the network.
		\subsection{Structure and Testing}
		A network simulation typically produces output in a GUI such that aspects of the network can be interpreted visually, such as to see how nodes interact, how data is sent, where connections go out of range or experience interference; it is possible to study the actual performance of a network and its protocols against the conceptual design. The simulation must be careful to provide an adequate level of detail to test the network without affecting the performance \cite{ leebreslauetal2000}. 
	\section{Routing}
		\subsection{Physical Routing}
		In order for mobile sensor networks to gather data about the environment, they will be required to navigate freely using predetermined pathfinding algorithms. For a drone network, a drone will be required to navigate 3-D airspace and collect sensing information, whilst being careful to maintain an efficient route and avoiding problems such as collision with its neighbours or the limitations of its physical components, such as battery life. Pathfinding must take into account the possibility of nonlinear dynamics, various constraints and changing environments \cite{robertsivillietal2012}.
		\subsection{Communications Routing}
		One of the core aspects of a sensor network is the ability for nodes to communicate with each other, relaying data back to the gateway. For an optimal routing algorithm, the exchange of data must be robust, avoiding congestion and maintaining connectivity when faced with mobility, whilst trying to maximise the duration for which the sensing task can be performed \cite{ curtschurgers2010}. The properties of communications routing which are to be optimised are dependent upon the type of sensor network; a drone network with less than thirty minutes of battery life must be optimised for energy consumption.

\chapter{Specification}
	In this section of the report, there will be an introduction to the possible problem(s) which are solvable through consumer-level drone networks, as defined in the previous section. After describing the problem, the objectives which need to be achieved in order to provide a solution for the problem space will be clearly laid out, to outline the foundation of the project. Justification for why the project solution to the aforementioned problem is both necessary and valid will also be given. There will be an analysis of the stakeholders in the project, followed by a feasibility study, where we provide a brief discussion of the scope of the problem which is to be handled, including the level of depth with which drone networks will be explored and implemented throughout the project. \\
This study also allows us to identify the possible problems which may arise and analyse the economic implications of the project, as well as give a brief introduction to the management of the project, in order to show that the project is actually feasible to complete with the time and resources available. Having defined the objectives which must be completed to provide a solution for the project, the subsequent functional and non-functional requirements must be identified, to ensure that the project deliverables are measurable and well-defined. Finally, any changes from the original specification at the beginning of the project will be briefly discussed, with justification for these changes.
	\section{Description of the Problem}
	The problem, in the context of this project, is that there are many situations in which humans are unable to effectively carry out a task, or would be put at risk, and so the situation is too difficult or dangerous to handle. In such situations, a well implemented drone network could not only perform better than a human could in a risk-free environment, but the use of drones as a sensor network provides a better alternative to other possible sensor network solutions, which will be discussed later. We can define an arbitrary problem which is impossible or risky for humans such as search and rescue, or high altitude photography, as well as problems which are feasible for humans, but can be more readily solved by UAVs, such as automated delivery. \\
	For this project, we have decided to focus on the possibility of using a drone network to detect and combat forest fires for the use of emergency services. While problems such as these have existed for a long time, solutions offered through sensor networks using drones is an area which is still in the early stages of research and development. While alternative solutions to using drone networks already exist, it is possible that the use of drones can expand upon existing solutions. Furthermore, the implementation of a drone network can hypothetically be applied to any similar problem area by altering the drone’s sensory inputs and outputs arbitrarily, and so a solution can be provided for the general use case, and applied to various other situations.
The solution to this problem can be split into four main areas: the network simulation, the physical routing, the communications routing, and the physical deployment. In order to successfully design and implement a drone sensor network, it is necessary to construct a stimulator which will accurately model the performance of the drone network, with associated physical and communications routing algorithms for optimal performance, before finally transferring this model to the physical drones and deploying them. The general solution can then be tested in a real situation, and then possibly extended to handle specific problem-solving tasks with user input. 
\section{Objectives}
\label{sec:obj}
The aim of this project, then, is to design and implement a general-use drone sensor network in order to solve the arbitrary problem of detecting and counter-acting forest fires based on user input, with the capability to be extended to any potential situation in the problem space. The major components of the project were outlined in the project specification during the early stages of the project life cycle, and can be summarised as follows: \\
(**THIS SECTION PROBABLY NEEDS WORK**) \\
(in the sense that tasks could be reordered, bullet points could be added for each enumerated item for more detail, tasks could be added/removed/reworded)
\begin{enumerate}
  \item \textbf{Implement a network simulator}. Either using a pre-existing set of libraries, or by creating our own, which are specifically tailored to our domain
  \item \textbf{Establish a network of drones with a base station}. Drones can communicate with one another and the base station, sending data between them
  \item \textbf{Provide autonomy to drones}. Each drone must be able to dynamically control itself, as opposed to remote control, in order to operate autonomously in a network
\item \textbf{Implement drone pathfinding}. Drones must carefully navigate through an area of physical space using well-defined rules for physical routing
\item \textbf{User input tasking}. The user must be able to define a problem for the network to detect, which is passed from the base station to the drones
\item \textbf{Implement problem detection}. Drones use sensory information to collect and pass data, notifying the base station in the event of a problem
\end{enumerate}
These tasks are roughly estimated to take an equal amount of time, where separate tasks can be assigned to each group member for maximum efficiency. Certain tasks, such as communication and pathfinding can be developed in parallel. Delegation of tasks and group roles will be discussed in depth in later sections.
	\section{Justification}
	The benefits of carrying out a project involved in creating an efficient, risk-free solution to emergency situations is relatively self-explanatory. The project is justifiable in its ability to produce potentially life-saving results and improve the general quality of human life, by implementing a consumer approach to any of the common usages that drones can be applied to and more, through the use of sensor networks. Drone networks themselves are an area of research which is growing in popularity, so the project interacts well with the state-of-the-art, and could have considerable impact on future developments. A general use solution for the project would be easy to use and deploy, and incredibly extensible.
	\section{Stakeholder Analysis}
	As previously discussed, the implementation of a drone sensor network has implications for a wide variety of industries. This section will therefore give a formal outline of the prospective stakeholders in the project and the justification for them. Firstly, with respect to detection and response to forest fires, emergency services such as firefighters, ambulance services and search and rescue would benefit greatly, by providing the ability to pre-empt danger and respond to crises much more rapidly, as well as reducing the risk to human lives in combating fires. Given that drone sensor networks are a relatively new field of research, those parties interested in research and development of sensor networks would also benefit from the project, as the results may extend the field of research, and the project solution could be adapted for use in other areas. In the same way, commercial businesses such as real estate and delivery services in pursuit of more efficient business may also benefit from the project. Finally, considering the use of drones in military operations, there are possible ramifications for military usage of the project, which will be discussed in further detail in later sections.
	\section{Feasibility Study}
	While the merits of the project are justifiable, it should also be noted that the project implementation carries a considerable technical difficulty, involving adaptation of individual, consumer-level drones to programmable, autonomous drones which function as a sensor network capable of algorithm-based communications and pathfinding, which are to be simulated and then physically deployed. Therefore, it is important to analyse the feasibility of the project, given time, hardware and other constraints, which will be discussed in the following sections.
		\subsection{Problem Scope}
		It is important to consider the scope of the problem, with regards to how far the solution can be extended into the domain of, in this case, search and rescue. Given that drone sensor networks are a relatively new domain for research and development, there is no commonly used and accepted standard for consumer-level drone networks in the context of search and rescue. In other words, the solution to the problem is not an extension of a previously existing solution, or of a set of rules governing how the problem can and should be solved using drone sensor networks.
As a result, the scope must be carefully defined such that an effective solution can be reached for the problem, without expanding too far into the problem domain. By implementing a drone sensor network which can be adopted into any general use case, which can be easily extended to take any required sensory information and applied to solve a problem, we can avoid overextending the project. At the same time, in order to show that the network can effectively handle user input tasking, we focus on taking one piece of sensor information, such as thermal, to demonstrate an accurate model for problem detection (as defined by the user), and response through well-established communications. 
		\subsection{Project Scope}
		Having defined the scope of the problem, it is also necessary to examine the scope of the project itself in terms of how far we extend into the domain of drone sensor networks. The implementation of any sensor network requires a lot of concise testing and a solid formation of protocols. There are many areas to consider with physical routing and communications, as well as physical deployment and use tasking. When creating the network simulator, we can tailor it to our specific requirements in order to minimise the amount of considerations for network protocols and possible problems, which will be discussed later, whilst accurately modelling real, physical deployment. 
Considering the time constraints that are imposed on the project, it will also be necessary to adapt pre-existing algorithms to establish our network communications and physical routing, as opposed to creating our own algorithm. In terms of physical deployment, the project is limited to two drones, so it is not possible to implement a full-blown network to test the solution. Nonetheless, it is possible to show that the network is theoretically scalable and accurately demonstrates the ability for drones to communicate and use pathfinding effectively.
		\subsection{Financial Analysis}
		The basic requirements for a physical drone network are the drones themselves, the sensors which will be attached to the drones, as well as equipment for programming the drones and communications. All of these are provided for free by the Department, so there are no financial constraints for physical deployment. While it is possible to purchase more drones, the same result can be achieved, provided that we have at least two drones. Additionally, the project will not include the use of any bespoke software or libraries in the development of the simulator or communications; they will be open source, so there are no costs incurred in the development of the project.
		\subsection{Market Analysis}
		For the project to be used by third parties, there must be licensing associated with the final product. As the project is intended to be open source and not commercialised, we will be using the GNU General Public License v3.0, which stipulates that our project is open source, with the freedom to use or change the software, as well as distribute it and share changes. However, in the event that the project is used, the software creators are not accountable for any problems with the project software. As the project is not commercial, with no intention to monetise it, the use of this license allows us to appease any and all stakeholders whilst protecting ourselves from potential threats.
		\subsection{Project Management}
		There are many challenges associated with undertaking a project such as this, particularly with regards to administration, scheduling and the division of tasks. Therefore, it is crucial that the project is managed efficiently, and that progress is carefully monitored and assessed throughout the life-cycle of the project. The issues associated with project management and how they were handled throughout the project will be discussed during the evaluation stages of the project. However, there are several challenges that the project group will be presented with in the context of project management, which are summarised below:
\begin{itemize}
\item \textbf{Development methodology}. There must be consideration for a development methodology which accommodates the size of the group and the time each group member has available, as well as how to track the progress of the project
\item \textbf{Deadlines and scheduling}. Meetings and deadlines must be scheduled such that the group is fully aware of what stage of the project must be completed and when. Interested parties such as the project supervisor must also be regularly informed of the progress of the project to allow for appropriate advisory actions where necessary
\item \textbf{Group roles}. Each member of the group must have a clear and distinct role in the group, with one member taking the role of project manager, dedicated to handling each group member and their associated tasks, as well as managing the state of the project to ensure that it is completed effectively and within constraints
\item \textbf{Managing project materials}. It is necessary to manage the project material to ensure consistency amongst each members’ work, as well as maintaining previous iterations of code to avoid problems such as losing previous functionality
\end{itemize}
	\section{Requirements Identification}
	This section will contain a list of the functional and non-functional requirements identified at the beginning of the project which will be necessary to implement to ensure that the project software adequately represents a solution to the problem as defined in this chapter. A functional requirement refers to quantitative requirements which can be easily measured and tested to ensure correct functionality. Non-functional requirements are those which cannot be measured, as they are subjective, but are nonetheless important for a successful project outcome. 

\newpage
(**THIS SECTION PROBABLY NEEDS WORK**) 
		\subsection{Functional Requirements}
\begin{table}[h!]
\centering
%\caption{My caption}
%\label{my-label}
\begin{tabular}{|l|l|}
\hline
\#  & Description                                                                                                                                                                                               \\ \hline
R1  & \begin{tabular}[c]{@{}l@{}}User must be able to run an executable which will set up the\\   simulation environment and communications module\end{tabular}                                                 \\ \hline
R2  & \begin{tabular}[c]{@{}l@{}}A class must be instantiated for the nodes of the simulator (i.e.\\   drones and base station)  and a communications\\   module must be installed on each of them\end{tabular} \\ \hline
R3  & \begin{tabular}[c]{@{}l@{}}The simulation must run, such that each node begins to operate and\\   ‘travel’ within the environment\end{tabular}                                                            \\ \hline
R4  & \begin{tabular}[c]{@{}l@{}}Each node must be capable of sending, listening for and receiving\\   messages over Wi-Fi/radio\end{tabular}                                                                   \\ \hline
R5  & \begin{tabular}[c]{@{}l@{}}The environment must incorporate and handle multithreading for each\\   node\end{tabular}                                                                                      \\ \hline
R6  & \begin{tabular}[c]{@{}l@{}}The simulation must be able to model and be robust to interference\\   between message passing\end{tabular}                                                                    \\ \hline
R7  & \begin{tabular}[c]{@{}l@{}}The user must be able to see output from each node as messages are\\   sent and received\end{tabular}                                                                          \\ \hline
R8  & \begin{tabular}[c]{@{}l@{}}The user must be able to visualise the output of the simulation on a\\   graphical interface\end{tabular}                                                                      \\ \hline
R9  & \begin{tabular}[c]{@{}l@{}}The base station node must be able to take user input which can be\\   broadcast to other nodes\end{tabular}                                                                   \\ \hline
R10 & \begin{tabular}[c]{@{}l@{}}The simulation must be capable of communications from any\\   (preselected) algorithms\end{tabular}                                                                            \\ \hline
R11 & \begin{tabular}[c]{@{}l@{}}The environment and its respective nodes should stop running when a\\   ‘KILL’ command (message) is received\end{tabular}                                                      \\ \hline
R12 & \begin{tabular}[c]{@{}l@{}}Nodes must be capable of moving through the environment autonomously\\   using pathfinding algorithms\end{tabular}                                                             \\ \hline
R13 & \begin{tabular}[c]{@{}l@{}}Nodes must be able to recognise useful sensory information, and\\   broadcast the information back to the base station\end{tabular}                                            \\ \hline
R14 & \begin{tabular}[c]{@{}l@{}}The simulation code must be adaptable to the physical drones for real\\   deployment\end{tabular}                                                                              \\ \hline
\end{tabular}
\end{table}
\newpage
		\subsection{Non-functional Requirements}
\begin{table}[h!]
\centering
%\caption{My caption}
%\label{my-label}
\begin{tabular}{|l|l|}
\hline
\# & Description                                                                                                            \\ \hline
R1 & \begin{tabular}[c]{@{}l@{}}Connections between nodes must be stable and resistant to\\   interference\end{tabular}     \\ \hline
R2 & Passing of messages must be within a reasonable timeframe                                                              \\ \hline
R3 & \begin{tabular}[c]{@{}l@{}}The broadcasting of information from drones or user input must be\\   accurate\end{tabular} \\ \hline
R4 & The user must be satisfied with the output from the environment                                                        \\ \hline
R5 & The final code must be extensible for more specific use cases                                                          \\ \hline 
\end{tabular}
\end{table}

	\section{Project Deliverables}
	Throughout the course of the project, there are several deadlines that must be met for which the project will be required to deliver a product. At the beginning of the project, the project specification must be submitted, which introduces the project and its initial planning and methodology, which will be appended to the report for reference. At the end of Term 1, a poster presentation will be given by the project group to select supervisors and invigilators, and a live demo of the project software and the final report are delivered at the beginning of Term 3. The demo will display the proposed solution to the problem defined at the start of the project, and showcase the features of the working solution. Alongside the final report, the code for the project, including the simulation, communications routing, physical routing and deployment code will also be submitted.
	\section{Changes From Original Specification}
	In the original specification, it was stated that NS3, a set of network simulator libraries, would serve as the core implementation of the simulator, and that the code for physical deployment would then be built off of the simulator as a result. However, we have decided to instead build our own simulator, after discovering that NS3 is relatively bloated and difficult to use. The benefits to creating our own stimulator are that the code can then be directly translated onto the drones during deployment, as well as being directly tailored to the project domain. As a result, the functionality provided by NS3 must be manually generated, which may increase the complexity of the project. 

\chapter{Literature Review}
	\section{Existing Solutions}
\section{Social and Ethical Issues}
\section{Physical Routing}
\section{MANETS}
\section{Drones}	

\chapter{Design}
	\section{Methodology}
\section{System Architecture}
	\subsection{Network Simulation}
		\subsubsection{Fundamental Structure}
		\subsubsection{Environment}
		\subsubsection{Application Programming Interfaces (API)}
	\subsection{Communications Routing}
	\subsection{Physical Routing} 
	\subsection{Physical Deployment}
		\subsubsection{Libraries}

\chapter{Simulation Software}
	\section{Existing Software}
	\subsection{NS3}
	\subsection{NS2}
	\subsection{Some}
	\subsection{other}
	\subsection{stuff}
\section{Summary of Existing Software}
\section{Development Methodology}
\section{Code Structure}
	\subsection{The Environment}
	\subsection{Communication Modules}
	\subsection{Drone}
	\subsection{Base Station}
\section{Results}
\section{Optimisation}
\section{Review Against Original Objectives}
\section{User Manual}

\chapter{Physical Deployment}
	\section{Objectives}
The main objective of this part of the project was to enable the running of octoDrone simulations on real hardware. While at first it may seem odd that we are attempting to move from a simulated programs to real ones when the overall goal of octoDrone is to allow for the simulation of real programs on virtual hardware. In a sense this a decision that logically follows the move from real to simulated - once I have built a program within the simulator framework and proved that it operates as I expect, it is infuriating to then have to implement this using a different set of frameworks on real hardware. It necessitates rewriting the same software in a way which effectively prohibits simulated benchmarks insomuch as there is no guarantee that simulated programs would exhibit the same performance characteristics as their real world counterparts given that they may be written on top of different libraries and possibly even in different languages. To this end it makes sense that one should want to run the same program in the simulator as in a real deployment.

There were a number of options that were explored in terms of the actual components that were used for our example deployment, similarly with the ways in which we adapted the simulator software to make the process as seamless as possible. The overall goal was to obtain a minimal working example, given that how well the hardware we could acquire would interact with what we had made was an unknown quantity. It is important to distinguish our efforts to make the simulator operate on real hardware units and the actual deployment we carried out. The former is a part of the product that is octoDrone, and was done to a high, professional standard. The latter was implemented using the components which were available to us through the department and, as such, does not represent what would be expected of a deployment carried out in industry. With that said, it is a testament to the flexibility of the simulator that it is possible to coerce it to run on distributed hardware components that it was never intended to support.

\section{Equipment and Feasibility}
The first decision to make with regards to the example deployment was to determine the type of hardware we would be targeting. There were two drones made available by the Department of Computer Science for our use - a pair of Parrot AR 2.0 Power Edition units.

\begin{aside}
\textbf{Parrot AR 2.0 specifications\cite{parrotspecs}}
\begin{itemize}
\item 1GHz 32 bit ARM Cortex A8 processor with 800MHz video DSP TMS320DMC64x
\item Linux kernel version 2.6.32
\item 1Gbit DDR2 RAM at 200MHz
\item Wi-Fi b,g,n
\item 3 axis gyroscope, accelerometer, and magnetometer
\item Ultrasound sensors for ground altitude measurement
\item 4 brushless inrunner motors. 14.5W 28,500 RPM
\item 30fps horizontal HD Camera (720p)
\item 60 fps vertical QVGA camera for ground speed measurement
\item Total weight 380g with outdoor hull, 420g with indoor hull
\end{itemize}
\end{aside}

\subsection{Micro-controllers}
The noteable thing missing from the above specifications is the ability to program the quadcopter with custom code. In a commercial or industrial deployment of this type, one would expect to use drones that are capable of being programmed themselves. Unfortunately these models are rather expensive, and as such we had to find a way to make the drones autonomous without being able to change the software running on them. The obvious solution was to use a micro-controller to pilot the drone because it would be easy to run our own custom software and communicate with the drone via its WiFi network. Below is a short summary of the investigation we did into the two major micro-controllers aimed at the consumer market (and thus within budget).

\subsubsection{Arduino}

\subsubsection{Raspberry Pi}

\subsection{Inter-node Communication}
In addition to communicating with the drone they were paired with, nodes would need to communicate with each other in order to pass messages and communicate with the base station. There were a number of options here which varied from true to life to more synthetic conditions.

\subsubsection{Radio}
In a real scenario, all communication done between nodes would be done using uni or bi-directional radio links. This allows for the use of software and protocols aimed at mobile sensor networks to facilitate the saving of power. It is obviously beneficial to create an environment which is as close as possible to real conditions, but we felt that this was less applicable for our example deployment. When we considered radio in the context of octoDrone, the abstraction of the physical and data link layers for communications meant that programs would not be able to tap into the extra power afforded by using it. In addition to this, many commercially available radio modules are designed to be used with the Arduino, which we had elected not to use.

\subsubsection{WiFi}
The second option we investigated was IEEE 802.11. The advantages over radio links were clear - it was easier to set up and use in in terms of program code, with many common protocols working automatically thanks to support and drivers for the linux kernel. In addition to this, there are many easily available USB adapters that work out of the box with the Raspberry Pi and Raspbian (a spin off of the Debian Linux distribution). This comes at the cost of the ability to make as many decisions at the lower OSI levels, but on balance this would make the process of developing and testing the deployment much easier. We concluded at this point that WiFi was preferred as a communication method over radio.

\subsubsection{Ethernet}
At this point the net was cast wider, in an attempt to see if there were any other solutions to the problem that were no immediately obvious. At this point we realised that the assumption that the micro-controllers would have to be mounted on the quadcopters themselves was a false one. Not only would having the Raspberry Pi units situated on the ground make flight more stable by reducing weight, it would also mean that we could use Ethernet to connect nodes together. This also sidestepped the issue we were encountering where the model A Raspberry Pis that the Department had given us only had two USB ports. Using two of these for WiFi adapters would have left no room for a keyboard. While not an insurmountable problem, it would have made debugging in particular more difficult than it needed to be. For the reasons above, as well as the aforementioned lack of necessity for accurate real world deployment conditions (due to the lack of budget) we chose Ethernet as the means of connectivity between nodes.

\subsection{Sensor Data}
Next came the problem of how we would deal with nodes sensing the environment. This largely came down to the decision to install sensors on the quadcopters or to have this data simulated as it would be done in a simulated environment. 

\subsubsection{Installing Physical Sensors}
\subsubsection{Emulating Real Data}

\section{Adapting the Simulator Code}
\section{Results}
\section{Review Against Original Objectives}


\chapter{Demonstration Program}
	\section{Drone and Base Station Sample Implementation}
In order to demonstrate the capabilities of the octoDrone product, both through the simulation and physical drones, a sample set of code was written. This demonstrative drone network uses heat sensors on the drones to collect temperature data across a predefined area. This does, of course, draw an immediate link with the forest fires prevalent in California. Once the 'normal' temperatures for an area are monitored, it could be extended such that any unexpected change to this temperature might mean there is the beginnings of a fire. Once alerted, wardens could confirm or refute this premise using cameras on the drone.

\subsection{The Base Station}
In this network, as expected, the base station will perform the task of disseminating work between the drones. In addition to this, it will be responsible for collating the data from the drones as it is being collected. Firstly, however, the base station broadcasts its address to everything nearby so that the drones know who to send temperature data to. This avoids flooding the channels with broadcasts to everything. Following this, the base station waits for a set amount of time during which it will receive the addresses of the drones in the network. These addresses are then stored as a vector on the base station.

Once this has been done, the base station determines which sections of the defined area will be allocated to each drone. This is done simply by splitting the area 1-dimensionally along the x-direction, where each drone is responsible for all points in the y-direction as shown in figure \ref{fig:area_dissemination}. The base station then waits messages from the drones containing temperature information.

\subsection{The Drones}
Similarly to the base station, the first thing the drones do is broadcast their address to everything around, followed by waiting to receive the address of the base station. Before the drones then do anything, they wait for the areas that they are to collect data on. Once this has been received, the drones systematically travel to each allocated point in space, measuring the temperature and moving onto the next. The physical routing techniques used for this are described in section \ref{sec:physicalrouting}.

Each time a drone collects data from a point in the area, it sends that data back to the base station immediately, meaning the base station receives many small messages each containing a single data point. This was done for two reasons. Firstly, the base station can spread the work of combining the data over the course of the system execution. Secondly, this means that if one of the drones fails, then its previous work is not lost.

\subsection{What Does this Demonstrate?}
This sample application demonstrates that it is very much possible for a working drone network to be built using the simulation. It shows the drones working autonomously and separate from each other, as well as a base station performing a similar task as would be expected in a real-world application. While this system only provides simple functionality, it demonstrates the possibility for a wide array of applications using fault tolerance, efficient communication, and robust physical routing.


\chapter{Testing}
	In this chapter the different tests developed for the simulator are detailed and described. A brief description of what each test was intented to verify (with rationale) will be given, along with information on how the results of the test were interpreted, if appropriate. Please note the difference between ``basic'' programs and the Basic communications module. Every attempt has been made to be unambiguous in this chapter, but the underwritten sections will be easier to follow with this distinction in mind.

\section{Unit Testing}

\subsection{The Simulator}
In order to test the individual components of the simulator, a number of different messageable programs and simulations were created.

\begin{itemize}
\item Movement
This unit test instantiates a drone and instructs it to move in one dimension, two dimensions, and finally in three dimensions. It also attempts to turn the drone. If the resulting heading and position in 3D space of the drone is the same as the known correct value then the test passes.

\item Communications
This unit test instantiates two drones which uses the most basic communications module (that simply delivers every message received to every messageable). It then attempts to send a message from one drone to the other. If the message is received and it correct, then the test passes.

\item Noise Functions
This unit test instantiates two drones using the most basic communications module, and passes a noise function to the environment which will drop the first message sent and deliver the second. The sending program dispatches two messages, and if the receiving program only detects the second message then the test passes.

\item Visualisation
The visualisation unit tests are comprised of each of the above tests, but run with the environment variable for visualisation set to true. If the results pass visual inspection (verified against the visualisation specification) then the tests are considered a success.
\end{itemize}

\subsection{Communications Programs}
Each communications module that was developed had its own unit test written. This served the function of testing the individual features of the communications algorithm in turn. As most of the tasks performed by these programs were heavily interlinked with others, it was decided that they would be tested as a whole (by sending a message one can test a number of co-dependent features in succession). If all of the tasks described in the test program succeed then the test passes. As there are multiple parts of the implementation being tested by a single simulation the communication module instances are set to output diagnostic information during the entirety of the process. If at any point the test fails then it is possible to determine which component is broken from this output stream.

\subsubsection{Basic}
The unit test for the Basic communications module verifies that messages can be passed to the communications module, broadcast to other nodes, received, interpreted (in the case of the ``KILL'' message), and delivered.

\subsubsection{Basic Addressed}
In addition to the above checks, this unit test verifies that the Basic addressed module will deliver messages only to the node identified by the IP address to which it is sent (or rather that nodes will drop any messages which they receive but are not addressed to them).

\subsubsection{AODV}
The testing procedure for AODV is significantly more complicated than for the above. In addition to all of the previously mentioned functionality, the unit test for the AODV communications module verifies that:

\begin{itemize}
\item nodes send hello messages at regular intervals
\item nodes seeking to send data to a node without an active route generate and broadcast RREQ messages
\item nodes seeking to send data to nodes with an active route use that route
\item nodes receiving an RREQ for which they have an active route reply with an RREP for that route
\item nodes receiving an RREQ for which they do not have an active route forward that RREQ to their neighbours
\item nodes which receive an RREP for a route they requested record that route if it more efficient than their current active route for the destination node (or if it is the only active route they have for that node)
\item nodes which receive an RREP for the route that they requested on behalf of another node return the received route if it is more efficient than the current active route they have for the destination (or it is the only active route they have for the destination)
\item nodes record information about other nodes from received messages, including hello messages
\item nodes receiving a data packet for which they are not the intended recipient forward that message on to the correct next hop on their active route to the destination if they have one
\item nodes receiving a data packet for which they are not the intended recipient and for which they do not have an active route drop that packet and propagate an RERR packet the next upstream node on that route.
\item nodes receiving an RERR packet for one of their active routes propagate that RERR message the next upstream node on that route, if any
\end{itemize}

\subsection{Parrot version of the Simulator}
A single test file was devised for the Parrot library which tested two the two areas of functionality which had been adapted from the original simulator - movement and communications delivery (as opposed to communications modules). The program could be run in sink or source mode, with sink mode simply waiting for a message to be delivered, and source mode sending a message. As soon as the sink version of the program received a message it performed a series of movements to make sure that the drone moved correctly (under both translation and rotation). This program was run on two hardware units, and the test failed if the drone failed to move correctly (or failed to move at all).

\subsection{Demonstration Program}
TODO: Ben to complete

\section{System Testing}
The nature of octoDrone's architecture was such that many features did not need to be system tested. For example, if a program was proved to be correct when using the Basic Addressed communications module, the abstracted nature of messaging meant that it would be provably correct using any other addressed communications module. Being able to reason about the components of the project in this way saved a large amount of time during development.

Another such inference was drawn about the correctness of the Parrot library and the main simulator library. If the main simulator was shown to be correct for a given set of functionality (excluding communications and movement), then the same had to have be true about the Parrot version of the simulator. This was because they shared large portions of the code base.

For those parts of the codebase that did require integration testing, the sample progam that was produced to demonstrate the capabilities of the project software made use of all of the features that were unit tested. Because of this it was the best way to ensure that the different parts of octoDrone interfaced correctly. Because this test was verifying multiple assertions, it was important that diagnostic information was emitted with enough detail to identify which part of the software had failed in the event of the test not passing.

\section{Testing Against the Specification}
For testing against the original specification, the demonstration program mentioned above contained all of the requirements mentioned in the specification (indeed, it is the reason that those requirements are there). As such, this was used to ensure that all of the individual parts of the simulator interoperated correctly. In the case that the test failed, diagnostic information was emitted with enough detail to identify which part of the software had failed.


\chapter{Project Management}
	\emph{There are many challenges which arise when trying to carry out a project which must be thoroughly documented and accounted for during the planning stages and subsequent design and implementation. For example, there must be considerations for how the project software will be developed, how the team will be organised, how the tasks will be scheduled and the various tools which will be used throughout the project. This chapter will discuss specific areas of project management such as these, and how the project group overcame the various challenges which arose throughout the duration of the project.}

\section{Project Methodology}

\subsection{Project Summary}
As with any software development project, it is important to first outline the aims of the project, the resources available to achieve these aims, and ensure that the stakeholders in the project have been correctly identified before the project begins. This allows for the project to have a clear view of its objectives and how to reach them. Furthermore, justification for why and how the project should be carried out is necessary to ensure that each group member is satisfied with the direction of the project and their roles within the group. This has been discussed in greater detail in the specification chapter, but will be summarised in this section to outline how the project developed from a management perspective.

The project idea to work with drone sensor networks was unanimously accepted by every group member; with the hardware freely available from the Department, each member was enthusiastic to tackle the various technical challenges which would arise in terms of hardware and software. UAVs and their applications in a sensor network is a subject which is currently a hotbed for research and development, with many different applications. The project deliverable presented an opportunity to experience working on the state-of-the-art in terms of technology and research, with the possibility of a lasting impact on the field.
Using several Parrot AR drones, the objective of the group was to create a simulator for sensor networks which could initialise and control airborne nodes, allowing them to function autonomously, as well as take user input to perform a task. The platform was then to be deployed onto hardware available from the Department, including external computers and extra peripherals.

The project was decided to be made for general use case, but with a focus on making a tool which could be used to perform research and outreach activities. The project supervisor approved the project, and continued to give helpful advice and considerations to assist with the planning stages. After agreeing upon the project, considering the resources and selecting a specific use case to work towards, the following work involved researching current methods and designing a simulation for the network. Once the simulation environment had been completed, communications and physical routing would be incorporated into the design, before finally transferring the platform onto the real drones for physical deployment. Each stage of the project was carefully monitored (as described later in this chapter), and tasks were split evenly between the group members in order to ensure that the project was completed within time constraints. Monitoring also ensured that any potential problems were addressed as they appeared. In the final stages of the project, the project deliverables, including the necessary documentation, were finalised and brought to a conclusion in order to be delivered safely and in accordance with the objectives which had been established during the planning stages of the project.

\subsection{Software Development Methodology}
For a software-based project, it is important to consider the development methodology which will be adopted, as the lack of effective practices will lead to unpredictability, repeated error, and wasted effort. Develop methodologies range from a document-heavy, tightly-structured waterfall method, to a lighter agile method which focuses on code iterations and adaptability. In order to reflect the number of the members of the group, with fluctuating schedules and time available to work on the project, a highly flexible adapted version of the agile scrum development was adopted. Core agile principles such as frequent delivery of working software and the welcoming of continuous changing requirements, as well as close, daily cooperation between developers have been maintained \cite{robertmartin2003}. However, flexibility is a very important element in the project, as students who may have other external factors to consider, such as deadlines for other assignments, would find traditional, more structured methods difficult to maintain. Waterfall methods are also more appropriate for larger businesses with long-term projects, and so would be inefficient considering the scope of the project.

The flexible approach afforded by agile development allows for adaptive development, such that milestones have been identified, but with flexibility to reach them, and allow for changes to requirements. As a result of the project group structure, compared to traditional scrum methods, sprints are not as heavily constricted by time, and there is no emphasis on providing a demo to the stakeholders during the sprint review. Additionally, the size of the development team, traditionally ranging from 3-9 people, has been downsized to two smaller teams of 1-3 individuals, and the daily scrum replaced with initiated sporadic bursts of development and discussion between members.

The scrum method is also appropriate because certain project tasks such as communications and physical routing can be completed simultaneously, such that the project software can be built up and tested modularly, and new features can be made use of immediately. This also allows for new iterations to influence and improve future iterations without extensive planning, which is suitable for a large project with many different considerations for each task in a small timeframe. Additionally, agile methods are preferred in a situation where an incremental delivery strategy based on rapid feedback is realistic \cite{iansommerville2010}, which is relevant to the structure of the project, with frequent meetings with the project supervisor and constant feedback available through correspondence regarding problems and requests.

When using an agile methodology, it is important to avoid the loss of information from a lack of documentation, which is common for a code-centric development methodology. This is especially dangerous for a development team with constantly changing members, as developments not being properly documented leads to a lack of knowledge and the inability to teach new members. However, this is a situation which does not apply to a static group, whose members will not change regardless of circumstance. 

\section{Team Structure}
As mentioned in previous sections, the project group consists of four members. With such a relatively small group, it was important to ensure that the workload was balanced equally between each member, and that all of the work was accounted for such that each and every task was being handled by one or more of the group members. A group consisting of fewer members is advantageous in that organisation is simpler; discussion and dissemination of tasks is much more straightforward, and each group member is aware of their assigned work, and any possible overlap with other group members. However, the lack of members results in an overwhelming disadvantage in terms of the amount of manpower available to work on the project at any given time. This also results in group members being assigned to work which they are unfamiliar with or are unqualified to handle.

As much as possible, each team member was assigned to a role which reflected their strengths in order to maximise efficiency for each task and simplify the scheduling process. The individual members and their assigned roles are shown below:

\begin{itemize}
\item \textbf{Alex Henson - Coordinating the report and research}. Responsible for the bulk of research in terms of drones, sensor networks, simulations and routing protocols, as well as handling the format and content of the report.

\item \textbf{William Seymour - Project manager, communications and physical deployment}. Responsible for distribution of tasks among group members and managing deadlines, implementation of communications  protocols and handling transfer of the project platform to the physical drones.

\item \textbf{Jon Gibson - Developing the simulator}. Responsible for the design and implementation of the drone sensor network simulator and the corresponding  foundational infrastructure and protocols for the real network.

\item \textbf{Ben de Ivey - Developing the demonstration simulation}. Responsible for utilising the simulator to create a simulation of the drone sensor network, as well as assisting with its development.
\end{itemize}

This list is not exhaustive, and many tasks were performed by more than one group member where necessary. The strength of the team could be seen in the ability to disseminate tasks easily and for each member to take responsibility for their individual tasks, using their skill sets to focus on the requirements most suited to them.

\section{Time Management}
One of the most important areas of project management is the ability to complete the project within time constraints. Therefore, it is necessary to accurately identify and schedule tasks to be completed within a reasonable timeframe. A projected timeline for the project is shown in figure \ref{timeline}, which outlines the fundamental tasks, and a Gantt chart is also included in figure \ref{gantt} which shows the scheduling of those events, from the beginning of the project until delivery. 

\begin{table}[]
\centering
\caption{Schedule of project tasks}
\label{timeline}
\begin{tabular}{|l|l|l|}
\hline
Time           & Task                                                                         & Details                                                                                                                                       \\ \hline
T1 W1,2        & \begin{tabular}[c]{@{}l@{}}Group Meeting\\   Supervisor Meeting\end{tabular} & Meet up to discuss project planning and schedule meeting times.                                                                               \\ \hline
T1 W3,4        & Project Planning                                                             & Decide upon tasks to be completed up until project specification.                                                                             \\ \hline
T1 W5,6        & Project Research                                                             & \begin{tabular}[c]{@{}l@{}}Research subject material to aid in design such as simulators and\\   drones.\end{tabular}                         \\ \hline
T1 W7,8        & Design Completion                                                            & \begin{tabular}[c]{@{}l@{}}Complete the design of each project component so as to begin\\   implementation.\end{tabular}                      \\ \hline
T1 W9          & Initial Project Specification                                                & \begin{tabular}[c]{@{}l@{}}Write-up project specification. \\   Have supervisor check draft and revise.\end{tabular}                          \\ \hline
T1 W9,10       & Simulation Environment                                                       & \begin{tabular}[c]{@{}l@{}}Complete implementation of basic simulation environment including\\   nodes and communication models.\end{tabular} \\ \hline
Holiday        & N/A                                                                          & Temporary buffer for tasks running overtime.                                                                                                  \\ \hline
T2 W1-5        & Communications Routing                                                       & Complete message passing using communications algorithms in simulator.                                                                        \\ \hline
T2 W1-5        & Physical Routing                                                             & Complete physical routing for nodes in simulator.                                                                                             \\ \hline
T2 W5-8        & Simulator Visualisation                                                      & Complete graphical output for simulation execution.                                                                                           \\ \hline
T2 W5-8        & Simulator Demonstration                                                      & \begin{tabular}[c]{@{}l@{}}Complete demonstration of drone network simulation to be adapted to\\   hardware\end{tabular}                      \\ \hline
T2 W9,10       & Physical Deployment                                                          & \begin{tabular}[c]{@{}l@{}}Complete adaption of simulation code to hardware.\\   Deploy network and record results.\end{tabular}              \\ \hline
Holiday, T3 W1 & Final Project report                                                         & Complete project report following results of deployment.                                                                                      \\ \hline
T3 W2-3        & Final Presentation                                                           & \begin{tabular}[c]{@{}l@{}}Prepared demo video for presentation.\\   Present results to complete the project.\end{tabular}                    \\ \hline
\end{tabular}
\end{table}

\begin{figure}
\centering	
\includegraphics[scale=1]{img/progantt.png}
\caption{Gantt chart of project schedule}
\label{gantt}
\end{figure}

\section{Progress Tracking}

\subsection{Meetings}
Regular meetings were scheduled once a week with both the project group and the project supervisor. Meetings with the supervisor consisted of sharing current progress, including any successes and challenges, as well as to seek advice on specific topics of difficulty such as simulation tools, accurately modelling error and routing techniques. Queries were also directed to the supervisor by email outside of meetings where necessary, to handle urgent requests or to check deliverables. 

The general group meetings involved all four members of the group gathering at the Department to discuss progress, assign new tasks as a result of discussions with the supervisor, or re-assign current tasks to improve scheduling. The group also examined the structure of the code, discussed bug fixes and new iterations of code, and corroborated any research findings using collaboration tools, which will be explained in section \ref{colab}. Extra meetings were also scheduled outside of the usual schedule in order to prepare for deadlines, or to coordinate workflow in the event of a task being assigned to more than one member. Certain tasks, such as physical deployment, required members to gather in one location in order to observe and work with the drones in person.

\subsection{Weekly Review}
In accordance with the software methodology chosen for the project, the group also informally discussed the status of the project in accordance with the schedule shown in figure \ref{timeline}, and received an update from each member as to their individual progress. By sharing the current state of each task with other members, it was possible to gauge the overall status of the project and to reassess any tasks which took up an inordinate amount of time. Instances of project review were also reserved for upcoming deadlines to ensure that project deliverables in particular were proceeding as planned. Minutes of meetings were kept in order to keep track of ongoing tasks.

\section{Source Control}
As a software-based project, it is important to ensure proper source control, so that code can be protected and members can be notified of new iterations, and how they differ from previous versions. While the source code itself did not have version numbers explicitly appended to it, Version Control Systems (VCS) were embedded into the collaboration tool used for code, including the ability to add messages which addressed the changes introduced by updated code, and to highlight any potential errors or bug fixes.

\section{Collaboration Tools}
\label{colab}

\subsection{GitHub}
To manage the project material, a git repository was created to ensure consistency amongst the project members' work and safety for the code base. Github, a web-based git repository hosting service, was used to incorporate source code management and distributed revision control to the project code. Each group member had their own repository locally, and branches were also used to ensure that major changes to the code base did not affect the original version of the code. This also allowed each member to avoid overwriting work when committing changes in the same area of code, which would result in merge conflicts. Using Github, the workflow was separated safely and manageably, and previous versions were retained by the repository so that they could not be lost. This also allowed each member of the group to contribute to the same repository without requiring them to be in the same physical space, or constantly exchanging updated code.

\subsection{Google Drive}
Google Drive was also used in order to manage project material, as well as have a backup for the code base and documentation. This is especially useful for project deliverables such as the report and the specification, as each member could collaborate in real-time with one another on the same documents. Google Drive also allowed for access on multiple devices such as phones, tablets and laptops, which was especially convenient for group meetings.

\subsection{Hastebin}
During the development of the project software, Hastebin was occasionally used for individual code snippets, as the UI supports code highlighting and is designed for sharing programming code. Alternatively, a real-time format such as Etherpad could have been used, but Hastebin was chosen for its ease-of-use and due to the fact that individual group members typically worked on separate areas of the same code, so Hastebin was utilised more for checking (incorrect) code than to develop code side-by-side.

\subsection{Facebook}
In order to communicate with other group members, Facebook was used as the primary method of contact. Each group member was already using Facebook, which made it the most common and the easiest way for group members to communicate. Facebook also allows for sharing files, which was found to be more convenient in the case of small files which did not require to be backed up. Facebook became a useful tool for general discussion about the project, including the project schedule and the code, as well as for the dissemination of tasks on-the-fly.

\section{Project Challenges}
At the beginning of the chapter, a summary was given of the main elements which the project was comprised of. In this section, the various challenges which had to be overcome in order to complete the project successfully are analysed. Specifically, areas such as researching, scheduling, organisation and development will be discussed, and how these challenges were dealt with throughout the project. 

\subsection{Planning}
The project was incredibly difficult in terms of the amount of work that had to be done in the limited amount of time available, and so effectively planning how and when each stage of the project would be completed was crucial to the project success. Each task which had to be completed was scheduled using the Gantt chart shown in figure \ref{gantt}, and the group was organised such that each member was aware of their individual tasks and responsibilities, and the timeframe available to them. The Gantt chart provided a good overview of the project activities, as well as being simple to read and understand; it was immediately clear that each member of the group what the current status of the project was, and what had to be completed at what time.

\subsection{Research}
One of the biggest challenges during the project was researching in order to find the proper tools and resources which were available that would be relevant to the project. A comparison of the various simulator libraries which were available was made, in order to find out their complexity, utilisation and functionality, such as ns-3 and OMNet++, and whether or not these libraries were necessary in order to create a drone sensor network. Ultimately, an original simulation was designed and developed by the group, which was influenced by the utilities provided by such libraries, and the use of ns-3 in the early stages of development was helpful to understand the various protocols which would need to be implemented into our own simulator.

\subsection{Development}
Following the planning stages, the project had carefully been broken down into tasks which had to be completed. According to the plan, the simulation would form the basis of physical deployment, in order to simplify the project as a whole and save time. During the development stage, the utmost care had to be taken to ensure that the simulation framework could be effectively transferred over to physical deployment during development. With this in mind, the simulation code was written in such a way that the only remaining factor to consider for physical deployment was the ability to use libraries which would interact with the drone's hardware, such as the ability to takeoff, move, turn, and land. Additionally, creating a simulation which could accurately model real-world communications without the use of network simulator libraries turned out to be a much more difficult task than had originally been scheduled for. The workload was subsequently re-evaluated and scheduled to realistically reflect the amount of time which would be taken, which involved using the time buffer of working over the holidays.

\subsection{Deployment}
When planning for physical deployment, there were several difficulties in terms of working with hardware which was not guaranteed to be suitable for the task. In order to provide autonomy to the drones, it was necessary to attach a Raspberry Pi to the drone, as well as GPS sensors, and then transfer the code base to the Pi in order to control the drone directly. Having no prior experience working with hardware explicitly, the group had to manage to configure the Pi in order to be able to connect and hook into the drone directly through Wi-Fi, and execute the code from the host (or base station). There were also concerns that attaching extra peripherals to the drone would potentially have weighed it down to the point that it would be incapable of sustaining its altitude, which would also affect its autonomous flight plan. The capabilities of the Parrot AR drone in terms of handling heavier payloads had to be researched and tested to confirm the maximum payload weight that it could handle.

\section{Risk Management}
The risks involved with the project are detailed in table \ref{risktable}.

\begin{table}[]
\centering
\caption{List of risks for the project}
\label{risktable}
\begin{tabular}{|l|l|l|}
\hline
\# & Risk Event                                                           & Mitigation                                                                                                                                                               \\ \hline
1  & Team member falls ill during the project                             & \begin{tabular}[c]{@{}l@{}}Reassign tasks where necessary and focus on schedule. Use buffer time\\   if necessary.\end{tabular}                                          \\ \hline
2  & Parrot drones have faulty hardware                                   & \begin{tabular}[c]{@{}l@{}}Reason with Department for replacement hardware and focus on other\\   tasks.\end{tabular}                                                    \\ \hline
3  & Any hardware breaks or is faulty                                     & Same as 2 - replace where necessary or continue without if possible.                                                                                                     \\ \hline
4  & Lack of contribution from a team member                              & \begin{tabular}[c]{@{}l@{}}Attempt to encourage the team member and assist with the task if\\   necessary. If problem persists, consult project supervisor.\end{tabular} \\ \hline
5  & Development turns out to be too difficult technically                & \begin{tabular}[c]{@{}l@{}}Consult project supervisor for advice as to changing the\\   project/project task/how to complete the task in question.\end{tabular}          \\ \hline
6  & Code becomes corrupted or lost due to accident/damage                & Recover backup code from git repository and continue.                                                                                                                    \\ \hline
7  & Simulation environment is too slow to perform as needed              & \begin{tabular}[c]{@{}l@{}}Consider optimisation of simulation as priority over other tasks in\\   order to preserve core functionality\end{tabular}                     \\ \hline
8  & Research for drone sensor networks is limited                        & \begin{tabular}[c]{@{}l@{}}Focus on combinations of alternatives e.g. research on drones,\\   research on sensor networks\end{tabular}                                   \\ \hline
9  & Unable to implement visualisation for the simulation                 & \begin{tabular}[c]{@{}l@{}}Focus on ensuring that other key functionality is achieved/maintained\\   for the simulator and consider as future work\end{tabular}          \\ \hline
10 & Drone gets lost during deployment due to loss of signal/out of range & Go call security lol                                                                                                                                                     \\ \hline
11 & Communications routing is too difficult/time costly to implement     & \begin{tabular}[c]{@{}l@{}}Consider finding and using pre-existing libraries which can be\\   imported to the stimulator\end{tabular}                                    \\ \hline
12 & Physical routing is too difficult/time costly to implement           & \begin{tabular}[c]{@{}l@{}}Consider reverting to core functionality or pre-existing libraries\\   where necessary\end{tabular}                                           \\ \hline
13 & Drones are unable to perform collision detection                     & \begin{tabular}[c]{@{}l@{}}Ensure that drones manage individual, well-separated airspace only\\   and are physically unable to go near each other\end{tabular}           \\ \hline
14 & Physical deployment is not completed on time                         & \begin{tabular}[c]{@{}l@{}}Focus on functionality of the simulator for demo and present\\   theoretical results for real deployment where possible\end{tabular}          \\ \hline
15 & Scheduling is poorly managed and tasks are overestimated timewise    & \begin{tabular}[c]{@{}l@{}}Re-evaluate schedule and use buffer time if necessary. Tasks may also\\   be added or removed where necessary\end{tabular}                    \\ \hline
16 & Demonstration code for simulation is not completed on time           & \begin{tabular}[c]{@{}l@{}}Revert to using example code used to test simulation for demo and\\   physical deployment.\end{tabular}                                       \\ \hline
17 & Use of drones on campus is deemed illegal/unsanctioned               & \begin{tabular}[c]{@{}l@{}}Find open airspace in alternate location which is not monitored/free\\   to use\end{tabular}                                                  \\ \hline
\end{tabular}
\end{table}

\chapter{Project Outcome}
	All in all the reference deployment of octoDrone was a great success. The final version of the parrot library was able to take simulator programs and run them on hardware with no changes and simulation files were usable with minimal changes. The quadcopters performed very similarly to our simulations, which simultaneously validates the accuracy of both the simulator and the deployment.

Running distributed applications on the Raspberry Pi was made extremely easy by having the compiled simulation take care of starting and stopping any additional threads it needed to create. This ease of use was such that it piqued the interest of a number of faculty who identified it as being an excellent outreach resource.

The results we managed to achieve also highlight how easy it would be to create a similar implementation for another hardware setup. In theory, it may also be possible to create a simulation which ran on a mixture of simulated and real drones. 

\chapter{Conclusion}
	As a project, the research and development of a generic simulator for implementing drone sensor networks has been incredibly challenging and rewarding. In the previous project evaluation chapter, considerations were made for the success of the project in terms of the components required and how the project was scheduled over the seven month period of the course. This chapter will give an overall outline of the project as a whole, including a summary of the project idea, the design and implementation of project components, the issues which were encountered along the way, and the outcome of the project in its completion. 
The project group, comprised of four people, were able to become acquainted and solidified themselves as a team in order to successfully overcome the tasks and challenges that came with the project. From the very beginning of the project, the team were able to establish a strong working atmosphere and an appreciation for the project concept, and the individual skills that each member would bring to the table in order to see it fulfilled. The project itself came with many hardships; drones sensor networks are a recently emerging field, which meant that it was difficult to find a large amount of material as a topic of research. The idea to create a sensor network with the use of drones in and of itself was a relatively new idea which required many considerations in terms of the feasibility of hardware, software libraries, and the existence of previous solutions to draw inspiration from. The prospect of designing and creating a simulator which could then serve as the framework for the deployment of a physical network of drones was incredibly exciting, and reflected the enthusiasm of each member to work on state-of-the-art technology (and do something really cool with drones). 
The creation of the simulator was originally designed to be based on the pre-existing network simulator ns-3, which is currently the most popular set of libraries for creating a network simulation. As originally laid out in the specification, the project was designed to use ns-3 as the foundation of the network simulator, to aid in the implementation of code which would be transferred to the hardware for deployment. After making the decision to change the basis of the project design by creating a network simulator in C++ from scratch, the group found that it was possible to create a general use simulator which could be adapted to any hardware more readily, and the software components developed into a platform which reflected the objectives of the project more accurately. The design for the simulator environment itself was broken down into its core components: the nodes in the network (drones and base station), the communications modules installed on the nodes, the routing protocols which would be used by the network and the code which would instantiate the environment and execute the simulation. The code for each component was meticulously planned, written, and tested, which was possible due to the modularity of the system, before being migrated into the simulator environment. 
There were many issues with the development of the simulator such as having to implement communications protocols manually, instead of being able to use pre-existing libraries which allow these protocols to be used automatically. While the idea to create a generic simulator from first principles was unique, it presented many challenges in practice, as the level of functionality which could be expected from using a simulator such as ns-3 would not be immediately available or feasible to introduce given the manpower and time scale of the project. However, this decision allowed for the group to create a platform which was simple and easy to manipulate – it contained only the bare necessary functionality for the task at hand, reducing the complexity and computation time significantly. Despite the technical difficulty involved, the group was able to successfully build a platform which could simulate a drone sensor network and output a visualisation of the results.
After the completion of the simulator, the next step was to implement physical deployment by transferring the simulation code over to real hardware. This reflected the objective to move from simulated programs to real hardware whilst avoiding the necessity of rewriting the same software, which is the appeal of the generic stimulator that was created. The physical deployment stage of the project went incredibly smoothly with only minor changes to the original design, such as using Ethernet for internode communication following the realisation that the Raspberry Pis did not have to be attached to the drones specifically. Despite concerns from the group that the hardware would not be compatible with the simulation or each other, alternative solutions were found for any outstanding problems and the project was scheduled such that allowances could be made for any hindrances resulting from hardware issues.
Overall, the project was completed successfully due to careful and concise planning, the enthusiasm and determination of each group member to complete their assigned tasks within their allotted time, and the well-founded management of the project in terms of scheduling and development methodology. The project itself can be said to have made contributions to the world as a combination of multiple fields of research with implications for future development and improved research into the potential of drone and sensor network technology. This includes the decision of the Department to make use of this project as a demonstration to prospective students, with the aims of garnering interest in the pursuit of Computer Science and similar concepts which can be derived from this project.
In final conclusion, the project has been a fantastic learning experience with regards to the development of a large team project over an extended period and the difficulties which are associated with it. The results of the project show that the concept is both possible to implement and expand upon, and that, despite being a relatively new field of research and development, there exist possible applications for future projects. The support and advice from the project supervisor and among the members of the group was critical in reaching the level of success that the project achieved, and it will be exciting for each member of the team to see how similar projects in the field ‘takeoff’ in the future (huehuehuehuehue).


\begin{appendices}
\chapter{API Documentation}
\includepdf[pages={-}]{../documentation/latex/refman.pdf}

\chapter{Original Project Specification}
\includepdf[pages={-}]{../specification/spec.pdf}

\bibliography{biblio}

\end{appendices}

\end{document}
