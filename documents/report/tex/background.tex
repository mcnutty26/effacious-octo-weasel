This section will introduce the components which will be researched into that form the basis of the project. The aim of this section is to provide the reader with definitions for keywords which will appear on numerous occasions throughout the project, as well as helping to lead into a definition of the problem space and the objectives of the project as a result. 
	\section{Drones}
		\subsection{Definition}
		Unmanned Aerial Vehicles (UAVs), also known as drones, are aircraft which are either ‘piloted’, or perform autonomously using pre-programmed flight path and objectives \cite{chriscolejimwright2010}. Consumer-level drones are typically small in size, and take the form of quadcopters, which are multi-rotor helicopters with four rotors.  These types of drones are very lightweight, and powered by batteries; power consumption is almost completely attributed to fight time, although a modified drone will be required to exhaust power on its additional parts. We will be focusing on the use of these drones for the scale of this project.
		\subsection{Sensor Capabilities}
		Drones are typically equipped with cameras, as well as additional sensors, which vary depending on the type of drone, or its purpose. In the case of military drones, sensors such as multi-spectral targeting systems, night vision, infrared imaging and GPS are an absolute must \cite{usairforce2015}. However, mounted weaponry may also be included for direct warfare, unless the drone is designed specifically for intelligence, surveillance or reconnaissance, as power consumption is a primary concern, and must be limited. Military drones are controlled via satellite from a military base, although drones may also be controlled by Wi-Fi, radio or remote. Consumer-level drones have a wide range of usages, and the sensors that they require to accommodate these tasks are largely dependent on the price. It is possible to attach a large multitude of different sensors to drones; however the primary function of an aerial (consumer-level) drone is to collect high-quality imagery, often beyond the level of detail that the human eye can process.  These types of sensors include stereoscopic, thermal imaging, near-infrared and infrared, but other sensors such as thermal sensors and proximity sensors may also be used \cite{ questuav2015}.
		\subsection{Usages}
		Consumer-level drones have become increasingly popular in the past few years as they have become more affordable, capable and reliable to use. Due to their ability to capture high-quality imagery from impossible-to-reach locations, drones are incredibly useful in areas such as real estate, to take aerial shots of properties, or as a cheap alternative to huge, expensive helicopters for capturing news, such as high speed chases \cite{josephdussault2014}. Drones can also be employed for services such as delivery; there have been several initiatives for drone-based delivery of food or packaged goods by famous companies such as Amazon and Domino's Pizza \cite{marcusfaires2015}. Another possible usage for drones is in emergency services, such as the detection of forest fires, or search and rescue. It is these areas of drone research and development which can be considered to display the strength and importance of drones, as they are able to perform dangerous tasks which humans are incapable of and/or with no physical risk to human beings themselves. Compared to other types of mobile sensing, drones offer direct control over where to sample the environment, such that they can be explicitly told where to move to.
	\section{Sensor Networks}
		\subsection{Definition}
		A wireless sensor network (WSN) is a wireless network consisting of spatially distributed autonomous devices using sensors to monitor physical or environmental conditions. These devices are referred to as nodes, which are able to communicate with each other and with a base node, commonly referred to as a gateway, which provides connectivity between itself, the nodes and the rest of the wired world \cite{ nationalinstruments2012}. Nodes may vary in size and number depending on the network, but will typically contain transceivers, a battery, an electronic circuit for interfacing with sensors and an energy source.  The ability to cooperatively pass sensor data to a main location has implications in many different industries, as well as military applications.
		\subsection{Drone Networks}
		In the context of drones, this refers to a wirelessly connected network of autonomous drones with a base station, which can distribute information or commands to the drones in the network, as well as facilitate communication between them and itself.  Given that autonomous drones are emerging as a powerful new breed of mobile sensing system which can carry rich sensor payloads with various methods of control, a collaborative network of drones has considerable potential, and can greatly extend the capabilities of traditional sensing systems \cite{lucamottola2014}. 
	\section{Network Simulation}
		\subsection{Definition}
		As the name suggests, network simulation is a technique for modelling the behaviour of a network without performing a real, physical deployment, in order to test the effectiveness of the network, and assess how the network will behave under different conditions. Therefore, a simulation refers to software that predicts the behaviour of a network, so that performance can be analysed. By emulating an existing network, unexpected problems can be addressed or prevented prior to the deployment of the network.
		\subsection{Structure and Testing}
		A network simulation typically produces output in a GUI such that aspects of the network can be interpreted visually, such as to see how nodes interact, how data is sent, where connections go out of range or experience interference; it is possible to study the actual performance of a network and its protocols against the conceptual design. The simulation must be careful to provide an adequate level of detail to test the network without affecting the performance \cite{ leebreslauetal2000}. 
	\section{Routing}
		\subsection{Physical Routing}
		In order for mobile sensor networks to gather data about the environment, they will be required to navigate freely using predetermined pathfinding algorithms. For a drone network, a drone will be required to navigate 3-D airspace and collect sensing information, whilst being careful to maintain an efficient route and avoiding problems such as collision with its neighbours or the limitations of its physical components, such as battery life. Pathfinding must take into account the possibility of nonlinear dynamics, various constraints and changing environments \cite{robertsivillietal2012}.
		\subsection{Communications Routing}
		One of the core aspects of a sensor network is the ability for nodes to communicate with each other, relaying data back to the gateway. For an optimal routing algorithm, the exchange of data must be robust, avoiding congestion and maintaining connectivity when faced with mobility, whilst trying to maximise the duration for which the sensing task can be performed \cite{ curtschurgers2010}. The properties of communications routing which are to be optimised are dependent upon the type of sensor network; a drone network with less than thirty minutes of battery life must be optimised for energy consumption.