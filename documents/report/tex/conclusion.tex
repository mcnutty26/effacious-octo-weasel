\emph{This chapter will give an overall outline of the project as a whole, including a summary of the project idea, the design and implementation of project components, the issues which were encountered along the way, and the outcome of the project in its completion.}

\section{Project Idea and Management}
As a project, the research and development of a generic simulator for implementing drone sensor networks has been incredibly challenging and rewarding. In the project evaluation chapter (chapter \ref{projeval}, considerations were made for the success of the project in terms of the components required and how the project was scheduled over the seven month period of the module. 

The project group, comprised of four people, was able to become acquainted and solidify themselves as a team in order to successfully overcome the tasks and challenges that came with the project. From the very beginning of the project, the team were able to establish a productive working atmosphere and an appreciation for the project concept, and the individual skills that each member would bring to the table in order to see it fulfilled. The project itself came with many hardships; drones sensor networks are a recently emerging field, which meant that it was difficult to find a large amount of material as a topic of research. The idea to create a sensor network with the use of drones in and of itself was a relatively new idea which required many considerations in terms of the feasibility of hardware, software libraries, and the existence of previous solutions to draw inspiration from. The prospect of designing and creating a simulator which could then serve as the framework for the deployment of a physical network of drones was incredibly exciting, and reflected the enthusiasm of each member to work on state-of-the-art technology. 

\section{Network Simulator}

The creation of the simulator was originally designed to be based on the pre-existing network simulator ns-3, which was the most popular application for creating a network simulation. As originally laid out in the specification, it was planned to use ns-3 as the foundation of the network simulator. After making the decision to change the basis of the project design by creating a network simulator in C++ from scratch, the group found that it was possible to create a general use simulator which could be adapted to any hardware more readily. The software components then developed into a platform which reflected the objectives of the project more accurately. The design for the simulator environment itself was broken down into its core components: the nodes in the network (drones and base station), the communications modules installed on the nodes, the routing protocols which would be used by the network, and the code which would instantiate the environment and execute the simulation. The code for each component was meticulously planned, written, and tested, which was possible due to the modularity of the system, before being integrated into the simulator environment. 

There were many issues with the development of the simulator such as having to implement communications protocols manually instead of being able to use pre-existing libraries which allow these protocols to be used automatically. While the idea to create a domain specific generic simulator from first principles was unique, it presented many challenges in practice, as the level of functionality which could be expected from using a simulator such as ns-3 would not be immediately available or feasible to introduce given the manpower and time scale of the project. However, this decision allowed for the group to create a platform which was simple and easy to manipulate - it contained only the bare minimum functionality for the task at hand, reducing the complexity and computation time significantly. Despite the technical difficulty involved, the group was able to successfully build a platform which could simulate a drone sensor network and output a visualisation of the results.

\section{Physical Deployment}

After the completion of the simulator, the next step was to implement an example physical deployment by transferring the simulation code over to real hardware. This reflected the objective to move from simulated programs to real hardware whilst avoiding the necessity of rewriting the same software, which is the appeal of the generic stimulator that was created. The physical deployment stage of the project went incredibly smoothly with only minor changes to the original design, such as using Ethernet for internode communication following the realisation that the Raspberry Pis did not have to be attached to the drones specifically. Despite concerns from the group that the hardware would not be compatible with the simulation, alternative solutions were found for any outstanding problems and the project was scheduled such that allowances could be made for any hindrances resulting from hardware issues.

\section{Reflections}

Overall, the project was completed successfully due to careful and concise planning, the enthusiasm and determination of each group member to complete their assigned tasks within their allotted time, and the well-founded management of the project in terms of scheduling and development methodology. The project itself can be said to have made contributions to multiple fields of research, with implications for future development and improved research methods surrounding drone and sensor network technology. This includes the decision of the Department to make use of this project as a demonstration to prospective students, with the aims of garnering interest in the pursuit of Computer Science and similar concepts which can be derived from this project.

\section{Closing}

In final conclusion, the project has been a fantastic learning experience with regards to the development of a large team project over an extended period and the difficulties which are associated with it. The results of the project show that the concept is both possible to implement and expand upon, and that, despite being a relatively new field of research and development, there exist possible applications for future projects. The support and advice from the project supervisor and among the members of the group was critical in reaching the level of success that the project achieved, and it will be exciting for each member of the team to see how similar projects in the field ``take off'' in the future.