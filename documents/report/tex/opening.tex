\section{Key Words}
Autonomous Drones, Sensor Networks, Network Simulation, Communications Routing, Quadcopters.

\section{License}
This work is licensed under the Creative Commons Attribution-ShareAlike 4.0 International License. This means that you are free to:

\begin{itemize}
	\item \textbf{Share} - Copy and redistribute the material in any medium or format.
	\item \textbf{Adapt} - Remix, transform, and build upon the material.
\end{itemize}

for any purpose (even commercially) under the following terms: 

\begin{itemize}
	\item \textbf{Attribution} -  You must give appropriate credit, provide a link to the license, and indicate if changes were made. You may do so in any reasonable manner, but not in any way that suggests the licensor endorses you or your use.
	\item \textbf{ShareAlike} - If you remix, transform, or build upon the material, you must distribute your contributions under the same license as the original.
	\item \textbf{No additional restrictions} - If you remix, transform, or build upon the material, you must distribute your contributions under the same license as the original.
\end{itemize}

\begin{figure}[H]
	\centering
	\doclicenseImage
\end{figure}

\section{Acknowledgements}
We would like to thank a few people in particular for helping us through the vast amount of research and development that was undertaken as part of this project. Our project supervisor Arshad Jhumka has helped guide us, and the advice he gave us when we were considering changing the direction of the project was invaluable. In addition, campus security were very understanding when asked to retrieve one of the drones used for testing from the roof of the maths building, and spared us what could have been a lot of embarrassment and trouble. On a lighter note, we would also like to thank USB Man and Will's rubber duck, who both made development immeasurably smoother.

\section{Introduction}
Drones have exploded into the public conciousness in recent years, for reasons both bad and good. With companies such as Amazon using drones for delivery, and many studios using them for filming, it is clear that they will become and remain a part of everyday life. Research into networks of drones is still young, and requires a specific set of tools. This project aims to create a new application for simulating drone networks to aid both academic progress and outreach. With a focus on adaptability, it is hoped that octoDrone can be used as a domain specific testing tool which is generic enough to be deployable to most types of consumer and commercial hardware available.

This report will provide a comprehensive analysis of the project undertaken by our group. There will be a background summary of the key components in this field, as well as a discussion of the ongoing research, development, and production being carried out. An analysis of the potential problems for which a solution can be found in drone networks will be supplied, and justification given for the resulting aims and objectives of our group. The report will detail the design, implementation, and testing of the solution, including considerations for the management of the project. Finally, the project outcome will be evaluated, followed by a conclusion reflecting on the success of the project and considerations for future works.