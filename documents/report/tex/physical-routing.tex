\section{Drone and Base Station Sample Implementation}
In order to demonstrate the capabilities of the octoDrone product, both through the simulation and physical drones, a sample set of code was written. This demonstrative drone network uses heat sensors on the drones to collect temperature data across a predefined area. This does, of course, draw an immediate link with the forest fires prevalent in California. Once the 'normal' temperatures for an area are monitored, it could be extended such that any unexpected change to this temperature might mean there is the beginnings of a fire. Once alerted, wardens could confirm or refute this premise using cameras on the drone.

\subsection{The Base Station}
In this network, as expected, the base station will perform the task of disseminating work between the drones. In addition to this, it will be responsible for collating the data from the drones as it is being collected. Firstly, however, the base station broadcasts its address to everything nearby so that the drones know who to send temperature data to. This avoids flooding the channels with broadcasts to everything. Following this, the base station waits for a set amount of time during which it will receive the addresses of the drones in the network. These addresses are then stored as a vector on the base station.

Once this has been done, the base station determines which sections of the defined area will be allocated to each drone. This is done simply by splitting the area 1-dimensionally along the x-direction, where each drone is responsible for all points in the y-direction as shown in figure \ref{fig:area_dissemination}. The base station then waits messages from the drones containing temperature information.

\subsection{The Drones}
Similarly to the base station, the first thing the drones do is broadcast their address to everything around, followed by waiting to receive the address of the base station. Before the drones then do anything, they wait for the areas that they are to collect data on. Once this has been received, the drones systematically travel to each allocated point in space, measuring the temperature and moving onto the next. The physical routing techniques used for this are described in section \ref{sec:physicalrouting}.

Each time a drone collects data from a point in the area, it sends that data back to the base station immediately, meaning the base station receives many small messages each containing a single data point. This was done for two reasons. Firstly, the base station can spread the work of combining the data over the course of the system execution. Secondly, this means that if one of the drones fails, then its previous work is not lost.

\subsection{What Does this Demonstrate?}
This sample application demonstrates that it is very much possible for a working drone network to be built using the simulation. It shows the drones working autonomously and separate from each other, as well as a base station performing a similar task as would be expected in a real-world application. While this system only provides simple functionality, it demonstrates the possibility for a wide array of applications using fault tolerance, efficient communication, and robust physical routing.
