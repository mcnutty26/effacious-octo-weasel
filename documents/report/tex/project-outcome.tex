All in all the reference deployment of octoDrone was a great success. The final version of the parrot library was able to take simulator programs and run them on hardware with no changes and simulation files were usable with minimal changes. The quadcopters performed very similarly to our simulations, which simultaneously validates the accuracy of both the simulator and the deployment.

Running distributed applications on the Raspberry Pi was made extremely easy by having the compiled simulation take care of starting and stopping any additional threads it needed to create. This ease of use was such that it piqued the interest of a number of faculty who identified it as being an excellent outreach resource.

The results we managed to achieve also highlight how easy it would be to create a similar implementation for another hardware setup. In theory, it may also be possible to create a simulation which ran on a mixture of simulated and real drones. 